%% arara: clean: { files: [cv_egon_geerardyn.aux, cv_egon_geerardyn.bbl, phdthesis.acn, phdthesis.acr, cv_egon_geerardyn.out, cv_egon_geerardyn.blg, cv_egon_geerardyn.log, cv_egon_geerardyn.run.xml, cv_egon_geerardyn.bcf] }
% arara: vc
% arara: xelatex: { shell : yes, synctex : yes, options : "-8bit" }
% arara: xelatex: { shell : yes, synctex : yes, options : "-8bit" }

\documentclass[grids]{cv-egeerardyn}

\firstName{Egon}
\lastName{Geerardyn}
\tagLine{Electronics Engineer with a passion for IT}
\maritalStatus{not married}
\birthDate{15 April 1988}
\birthPlace{Jette (BE)}
\nationality{Belgian}
\address{Zomerstraat 26 GV 00\\1050 Brussels (Belgium)}
\email{egon.geerardyn@gmail.com}
\website{egeerardyn.github.io}
\phone{+32 472 65 61 08}

% \begin{profiles}
\profileGithub{egeerardyn}
\profileLinkedIn{egongeerardyn}
\profileResearchGate{Egon\_Geerardyn}
\profileStackOverflow{Egon}{514071}
%\profileFacebook{egon.geerardyn}
% \end{profiles}

\begin{document}

\maketitle
\sectionHidden{education}
\section{experience}
\section{languages}
\section{competences}


\end{document}
%%%%%%%%%%%%%%%%%%%%%%%%%%%%%%%%%%%%%%%%%%%%%%%%%%%%%%%%%%%%%%%%%%%%%%%%%%%%%%%%%%%%%%
%!TEX TS-program = xelatex
% \documentclass[]{friggeri-cv}
% \addbibresource{bibliography.bib}

% \begin{document}
% \header{adrien}{friggeri}
%        {social network analyst}


% % In the aside, each new line forces a line break
% \begin{aside}
%   \section{about}
%     31 rue Smith
%     69002 Lyon
%     France
%     ~
%     \href{mailto:adrien@friggeri.net}{adrien@friggeri.net}
%     \href{http://friggeri.net}{http://friggeri.net}
%     \href{http://facebook.com/adrien}{fb://adrien}
%   \section{languages}
%     bilingual french/english
%     spanish \& italian notions
%   \section{programming}
%     {\color{red} $\varheartsuit$} JavaScript
%     (ES5, node.js)
%     Python, C, OCaml
%     CSS3 \& HTML5
% \end{aside}

% \section{interests}

% complex networks, social networks, community detection, community structure,
% overlapping communities, information diffusion, viral marketing, social
% inference, recommendation, data mining

% \section{education}

% \begin{entrylist}
%   \entry
%     {since 2009}
%     {Ph.D. {\normalfont candidate in Computer Science}}
%     {DNET/INRIA, LIP/ÉNS de Lyon}
%     {\emph{A Quantified Theory of Social Cohesion.}}
%   \entry
%     {2007–2008}
%     {M.Sc. magna cum laude}
%     {IXXI, École Normale Supérieure de Lyon}
%     {Majoring in Computer Science\\
%     Specialization in Complex Systems}
%   \entry
%     {2006–2007}
%     {B.Sc. magna cum laude}
%     {École Normale Supérieure de Lyon}
%     {Majoring in Computer Science}
%   \entry
%     {2003–2006}
%     {Classes Préparatoires aux Grandes Écoles}
%     {Lycée Fénelon, Lycée Louis le Grand, Paris}
%     {Preparation for national competitive entrance exams to leading French ``grandes écoles'', specializing in mathematics and physics.}
%   \entry
%     {2003}
%     {French Baccalauréat S. with honors}
%     {Lycée Louis le Grand, Paris}
%     {Specialization in mathematics and physics}
% \end{entrylist}

% \section{experience}

% \begin{entrylist}
%   \entry
%     {02–07 2009}
%     {LIP6/CNRS, Paris}
%     {Research Internship.}
%     {\emph{Visualization of complex networks.}}
%   \entry
%     {06–08 2008}
%     {ISCPIF/CNRS, Paris}
%     {Research Internship.}
%     {\emph{Diffusion in the Blogosphere. Happy Flu.}}
%   \entry
%     {06–08 2007}
%     {LIP6/CNRS, Paris}
%     {Research Internship.}
%     {\emph{Kernels in real world networks.}}
%   \entry
%     {07–08 2005}
%     {\href{http://www.kelkoo.com}{Kelkoo.com}}
%     {Summer job.}
%     {\emph{Creation of a keyword generator for Google Adwords.}}
%   \entry
%     {07–08 2004}
%     {\href{http://www.monsieurprix.com}{MonsieurPrix.com}}
%     {Summer job.}
%     {\emph{Development of an e-commerce product indexation spider.}}
% \end{entrylist}

% \section{applications}

% \begin{entrylist}
%   \entry
%     {2012}
%     {Who did I forget ?}
%     {\href{http://whodidiforget.com}{whodidiforget.com}}
%     {Guest list recommendation for Facebook events based on friends already attending the event.}
%   \entry
%     {2011}
%     {Fellows}
%     {\href{http://fellows-exp.com}{fellows-exp.com}}
%     {Automatic community detection among Facebook Friends in order to validate the \emph{cohesion} measure, creation of friend lists.}
%   \entry
%     {2008}
%     {Happy Flu}
%     {\href{http://happyflu.com}{happyflu.com}}
%     {Experiment aimed to measure viral spreading of content across the blogosphere.}
% \end{entrylist}

% \section{publications}


% \printbibsection{article}{article in peer-reviewed journal}
% \begin{refsection}
%   \nocite{*}
%   \printbibliography[sorting=chronological, type=inproceedings, title={international peer-reviewed conferences/proceedings}, notkeyword={france}, heading=subbibliography]
% \end{refsection}
% \begin{refsection}
%   \nocite{*}
%   \printbibliography[sorting=chronological, type=inproceedings, title={local peer-reviewed conferences/proceedings}, keyword={france}, heading=subbibliography]
% \end{refsection}
% \printbibsection{misc}{other publications}
% \printbibsection{report}{research reports}

% \end{document}
